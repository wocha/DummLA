\section{Grundlagen}

\subsection{Normierte Räume)}

Sei $\mathbb{K}=\mathbb{C}$ ein Körper und $\mathbb{K}=\mathbb{C}$ ein Körper und $V$ ein Vektorraum über $\mathbb{K}$.

\subsubsection{Norm}
Eine Abbildung $\|\cdot\|:V\longrightarrow\mathbb{R}$ heißt Norm \textbf{Norm}, falls gilt:
\begin{align*}
	\|v\|  >  0  \forall  v  \in  V\backslash\left\lbrace 0 \right\rbrace \text{(Definitheit)} \\
	\|\lambda v\|  =  |\lambda|\cdot\|v\|  \forall\lambda  \in  \mathbb{K},\forall v\in V\text{(absolute Homogenität)}\\
	\|v+w\| \le \| v\| \| w\| \forall v,w \in V \text{(Subadditivität bzw. Dreiecksungleichung)}
\end{align*}

Standardbeispiel für eine Norm ist die \textbf{euklidische Norm} eines Vektors $(x,y)$ in der Ebene $\mathbb{R}^2$
\begin{align*}
	\|(x,y)\| = \sqrt{x^2+y^2}
\end{align*}

\subsubsection{Normierter Raum}
Ein Vektorraum $V$ zusammen mit einer Norm $\|\cdot\|$, geschrieben $(V, \|\cdot\|)$, heißt \textbf{normierter Raum}.