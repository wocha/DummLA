\section{Grundlagen der LA und der Fehlerberechnung}

\subsection{Normen)}

Eine Norm ist eine Abbildung $ \|\cdot \|$ von einem Vektorraum $ V$ über dem Körper $  \mathbb {K} $  der reellen oder der komplexen Zahlen in die Menge der nichtnegativen reellen Zahlen $  {\mathbb {R} }_{0}^{+} $,

$  \|\cdot \|\colon V\to {\mathbb {R} }_{0}^{+},\;x\mapsto \|x\| $,
die für alle Vektoren $  x,y\in V $ und alle Skalare $  \alpha \in \mathbb {K} $  die folgenden drei Axiome erfüllt:

(1) Definitheit:	$  \|x\|=0\;\Rightarrow \;x=0 $,\\
(2) absolute Homogenität:	$  \|\alpha \cdot x\|=|\alpha |\cdot \|x\| $,\\
(3) Subadditivität oder Dreiecksungleichung:	$ \|x+y\|\leq \|x\|+\|y\| $.\\
\\
Hierbei bezeichnet  $|\cdot |$ den Betrag des Skalars.\\
$(V,\|\cdot \|)$ heißt normierter Vektorraum.